%% LyX 1.3 created this file.  For more info, see http://www.lyx.org/.
%% Do not edit unless you really know what you are doing.
\documentclass[a4paper,oneside,german]{extbook}
\usepackage[T1]{fontenc}
\usepackage[latin1]{inputenc}
\setcounter{secnumdepth}{5}
\setcounter{tocdepth}{5}

\makeatletter
\usepackage{babel}
\makeatother
\begin{document}

\title{Ausfallskonzept ICM/Recall}

\maketitle
Serverraum\\
Beim Neustart der Rechner wird zuerst der Recall\_Srv, dann der Medidata-Rechner
gestartet.\\
Automatisch wird der HL7 Nachts neu gestartet.\\
\\
Finden sich in der Anzeige {}``unbearbeitete Messages'' muss HL7
neu gestartet werden.\\
\\
Der Medidata Server startet immer gegen 23:30 Uhr neu.Sollten Probleme
(wie fehlende Vital- oder Evitatdaten) auftreten, muss neu gestartet
werden.Nach dem Neustart ist nur die Medidata Shell offen.Das in der
Taskleiste befindliche Men� kann Ausk�nfte �ber die einzelnen Verbindungen
geben.\\
\\
Der RecallServer zeigt um die 60 \% Prozessorlast im Maximum an..Bei
Problemem im Men� Zubeh�r Den Ereignismanager aufrufen.\\
die perflib Problematik ist bekannt!!

Die Textdatei auf dem Desktop ist die Kommunikationschnittstelle mit
Dr�ger!!\\
\\
Der Admin PC 

Meldet eine Recall WS dass man nicht angemeldet werden kann, da die
WS gerade benutzt wird, muss man im Men� Programme Recall das Recall
Setup aufrufen\\
Login:System\\
PW: System\\
auf der Oberfl�che ist es m�glich die einzelnen WS zu entserren.\\
\\
Auf dem Admin PC sollten die 4 Recall Fenster laufen(AIMS Mailboxing,Recall
Interface,TCP/IP Interface und noch eines)\\
\\
Sollte es Schwierigkeiten geben ist der {}``Blaue Draht'' anzurufen.Der
Auftrag ist sehr dringend zu machen!!\\

\end{document}
